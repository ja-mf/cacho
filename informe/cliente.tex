\subsection{Cliente Javascript}
La parte del cliente se compone de los templates
de cada pagina HTML, junto a las hojas de estilo (CSS)
y scripts JS. Estos archivos se ubican en las carpetas
\texttt{templates} y \texttt{static} respectivamente. \\

En una sala de juego, se carga la biblioteca Socket.IO
(\texttt{socket.io.js}), sobre la cual se construiran manipuladores de eventos asincronicos de emisión y recepción de mensajes con el servidor.\\

Al gatillarse estos eventos, irán a producir cambios en el HTML (GUI),
ocupando la biblioteca JQuery para mayor facilidad.\\

El siguiente codigo muestra la implementacion (reducida) del cliente. Considerar lo siguiente:
\begin{itemize}
	\item \texttt{io.connect('/game');} conectará el cliente Socket.IO al \texttt{Namespace}
	\texttt{game}, definido anteriormente en \texttt{cacho\_socketio.py}
	\item \texttt{socket.on('connect', function (username) ...} manipulará la llegada del evento
	\texttt{connect}, con argumento \texttt{username}, luego enviará la peticion a unirse a la sala
	mediante \texttt{emit}
	\item \texttt{socket.on('usuarios\_room', ...} manipulará la llegada del evento \texttt{usuarios\_room}, la cual contendrá como argumento a una lista de usuarios con informacion acerca de ellos en JSON.
	\item En sintesis, el metodo \texttt{emit(event\_name, args)} enviará al servidor el evento \texttt{event\_name},
	conteniendo como información a \texttt{args}. El metodo \texttt{on(event\_name, callback)} ejecutará
	la funcion \texttt{callback} (con argumentos si es que posee mensaje) en la llegada del evento \texttt{event\_name}.
\end{itemize}

\begin{lstlisting}[language=JavaScript, caption=cacho\_app/static/js/cacho.js]
// socket.io specific code
// funciones de socket del cliente.
var socket = io.connect("/game");
var sessid;
var user_list;

var numeros = ['ningun', 'un', 'dos', 'tres', 'cuatro', 'cinco',
					'seis', 'siete', 'ocho', 'nueve', 'diez', 'once',
					'doce', 'trece', 'catorce', 'quince', 'dieciseis',
					'diecisiete', 'dieciocho', 'diecinueve', 'veinte']
var pintas = ['0', 'aces', 'tontos', 'trenes', 'cuartas', 'quintas', 'sextas']

socket.on('connect', function (username) {
   $('#chat').addClass('connected');
   socket.emit('join', window.room);
});

socket.on('user_sessid', function(id) { sessid = id; });
socket.on('turno', function(turno) { ... });
socket.on('usuarios_room', function (usernames) { ... });
...
\end{lstlisting}




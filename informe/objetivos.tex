\subsection{Objetivos}
\subsubsection{General}
El objetivo principal de la aplicacion es ofrecer una plataforma para
un juego popular chileno, en linea, activo y rapido. La lógica del juego
es de relativa sencillez y las reglas son ampliamente conocidas, por lo que
pretende que la jugabilidad sea alta.

\subsubsection{Especificos}
\begin{itemize}
	\item Promover el juego de dudo.
	\item Disponer de un juego rapido e instantaneo
	\item Utilizar tecnologías que permitan su jugabilidad de forma nativa,
	no es necesario descargar nada para empezar a jugar.
	\item Contar con un chat y salas de juego.
	\item Establecer un protocolo de comunicación entre el cliente y el servidor (definición de mensajes).
\end{itemize}

\subsection{Tecnologias utilizadas}
Para la parte del servidor, constará de una aplicacion MVC, utilizando el framework
Django (Python). Con el fin de tener una comunicación asincronica y activa entre
el cliente y el servidor, se utilizará un binding de la biblioteca Socket.IO (Javascript),
construido sobre gevent, una biblioteca basadas en co-rutinas para aplicaciones de red, llamado
gevent-socketio.\\

En el cliente se utilizará HTML5 y CSS para el diseño, y para la comunicación, Javascript
con la biblioteca Socket.IO, basada a su vez en node.js.\\

Se posee un repositorio git para el control de versiones, hospedado en github.com



